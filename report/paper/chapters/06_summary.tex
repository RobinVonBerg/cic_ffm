\chapter{Bewertung}
    Unsere Bewertung der Projektidee und des Prototypen fällt überwiegend positiv aus. Wir denken, dass das Konzept eines interaktiven und smarten Mülleimers an bestimmten Orten mehr Aufmerksamkeit auf sich ziehen und somit zur Reduzierung von Littering beitragen kann. Auch die Verwendung von Designaspekten die den Menschen im täglichen Leben begegnen und mit denen sie sich identifizieren können, trägt dazu bei, dass die Menschen ein solches Konzept annehmen werden. Zudem bietet der Designaspekt im Fall unseres Projekts ein Alleinstellungsmerkmal für die Stadt Frankfurt und unterstützt somit zugleich das Stadtmarketing.\\

    Der vorgestellte Prototyp wurde jedoch keinem öffentlichen Praxistest unterzogen. Um die getätigten Aussagen zu untermauern muss der Prototyp zu einem vollständigen Produktprototyp weiterentwickelt werden, welcher in einem Frankfurter Park, oder auf dem Frankfurter Opernplatz aufgestellt werden kann um die Verwendung und Wirkung in der Praxis zu beobachten.\\

    Zudem ist Anzumerken, dass interaktive und smarte Mülleimer nur an bestimmten Orten Eingesetzt werden können und sollten. Dazu zählen insbesondere öffentliche Plätze und Parks. Zudem ist uns bewusst, dass unser Produkt nur einen geringen Teil zur Reduzierung des Müllproblems leisten kann. Eine wesentlich wichtigere Zielsetzung sollte es sein, die Menschen, dazu zu bringen weniger Müll zu produzieren und ihnen klar zu machen, wie wichtig fachgerechte Müllentsorgung ist. Dies kann durch Bildung und Aufklärung erreicht werden und dabei kann unser Gerippter bestenfalls unterstützend wirken.


\chapter{Zukünftige Entwicklungsmöglichkeiten}
    Wie im Kapitel \ref{summary} erwähnt, wurde die Implementation eines Füllstandssensors vorbereitet, dieser jedoch aus Zeitgründen nicht in das Endprodukt integriert. Hier setzt sogleich die erste Möglichkeit zur Weiterentwicklung an. Der Füllstandssensor kann in den Mülleimer integriert werden und zusätzlich kann der Mülleimer mit dem LoRaWAN Netz der Stadt Frankfurt verbunden werden um darüber die verschiedenen Sensordaten zugänglich zu machen.\\

    Ein weiterer wichtiger Schritt zum ausgereiften Produkt ist die Integration einer netzunabhängigen Stromversorgung durch Akkus und Solarzellen, damit der Mülleimer an den vorgesehenen Standorten autark installiert werden kann, ohne vorher größere Umbaumaßnahmen an den Standorten vorzunehmen.\\

    Die wichtigste Weiterentwicklung ist jedoch vorerst die Entwicklung eines Modells aus beständigen Materialien, die in der Praxis eingesetzt werden können. In der Vorstellung des Projektteams lag dabei ein Rauten-Skelett aus lackiertem Metall, wobei die Rauten durch Plexiglas oder ein ähnliches Material ausgefüllt werden, wie es im kleineren Modell zur Präsentation der Bluetoothkommunikation angedeutet wurde.

\chapter{Repository}

    Unser gesamter Arduino Code, unsere Projekttagebücher, dieser Projektbericht und dessen Quellcode sowie weitere Materialien sind unter folgendem privaten Repository zu finden: \url{https://github.com/RobinVonBerg/cic_ffm}

    