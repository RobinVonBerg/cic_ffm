\chapter{Fazit}

\section{Was lief gut}
    Hervorzuheben ist, dass die initiale Zusammenarbeit im Team hervorragend funktionierte. Die unterschiedlichen Kompetenzen wurden klar kommuniziert und berücksichtigt, woraus sich ein sehr harmonisches und produktives Arbeitsumfeld ergab. Im Ideenfindungsprozess wurde großer Wert auf die interdisziplinäre Zusammenarbeit gelegt, so konnte jeder seine Kompetenzen in die Bereiche der anderen Studiengänge erweitern und selbst ein Teil seines Wissens weitergeben, was zu einer guten gemeinsamen Wissensbasis führte. Dadurch gab es im gesamten Projektverlauf wenige fachliche Kommunikationsbarrieren und wenn welche auftraten konnten sie leicht beseitigt werden.\\

    Des weiteren war die Ernennung einer geeigneten Projektmanagerin von großem Vorteil. Celina hat die Gruppe geführt, wenn Führung benötigt wurde, ohne dabei eine dominante Position einzunehmen. Insgesamt ist ihr Führungsstil sehr freundlich und offen, was ebenfalls sehr stark zum angenehmen Arbeitsklima in der Gruppe beigetragen hat. Zusätzlich zur hervorragenden Projektleitung hat das eingeführte wöchentliche Meeting inklusive kurzem Chek-In zu Beginn jedes Meetings, in dem jeder ein Update zum persönlichen Befinden geben konnte, das Gruppenklima positiv beeinflusst und dazu beigetragen, dass die Gruppenmitglieder sich, während längerer Phasen ohne Präsenztreffen, nicht aus den Augen verlieren. \\

    Ebenfalls positiv hervorzuheben ist die Zusammenarbeit mit der FES, beziehungsweise genau genommen mit Herrn Jochen Schmitz. Jochen hat eine Begeisterung an den Tag gelegt, die das Team motiviert und bei Laune gehalten hat und stand immer für etwaige Fragen zur Verfügung.\\

\section{Was haben wir gelernt}
    Das Ernennen von Verantwortlichen für bestimmte Teilbereich, insbesondere in der Prototyping Phase, hätte einige Vorteile mit sich gebracht. Der Verantwortlich sollte dabei jedoch nicht zwingend derjenige sein, der die Aufgabe selbst oder im Alleingang erfüllt, sondern das Kontrollorgan darstellen und überwachen, das bestimmte Teilaspekte zu bestimmten Deadlines erfüllt sind. Da solche Verantwortliche nicht benannt wurden, kam es oft zu Missverständnissen, was die Zeitpunkte für die Fertigstellung einzelner Komponenten anging.\\

    Zudem sind die Einblicke in die Denkweise fachfremder Personen in einem solchen Projekt ein großer Erkenntnisgewinn. Das Projekt hat gezeigt, dass das Schaffen gemeinsamer fachlicher Grundlagen essenziell ist für eine erfolgreiche Kommunikation im interdisziplinären Kontext.\\

    Weitere Lernaspekte liegen insbesondere im technischen Bereich, da wir beide Informatik studieren und mit der Einbindung von Sensorik und Aktoren in Software bisher keine Berührungspunkte hatten. Dadurch, dass wir in diesem Semester die Vorlesung zur Veranstaltung Interface-Technologie hören, konnten wir bereits erstes theoretisches Wissen in Bezug auf Sensorik sammeln und dieses nun durch praktische Anwendung in diesem Projekt vertiefen.\\

    Zusätzlich hat das Projekt verdeutlicht, dass die Komponentenauswahl für einen solchen Prototypen weitaus besser geplant werden kann und viel Zeit in Anspruch nimmt, da es mit der einfachen Auswahl aus einem Produktkatalog nicht getan ist. Es ist durchaus legitim verschiedene Komponenten für die gleiche Aufgabe in der Praxis zu testen, um Vor- und Nachteile auch in der Anwendung der Komponenten ausfindig zu machen. Eine Auswahl nur anhand von Daten aus einem Datenblatt zu treffen kann in der praktischen Verwendung der Komponenten durchaus zu ungeahnten Komplikationen und Mehraufwand führen.\\

    Zuletzt ist noch festzustellen, dass das Überwachen des Budgets unmöglich ist, wenn es zwar eine zentrale Stelle zur Überwachung des Budgets gibt, diese aber keine Information über voraussichtliche oder tatsächliche Kosten von Bestellungen erhält. Daraus resultiert, dass es wichtig ist Bestellprozesse vom Anfang bis zum Ende zu planen und dabei den korrekten Informationsfluss im Auge zu behalten.